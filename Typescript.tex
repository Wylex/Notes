\documentclass[french]{article}

\usepackage[utf8]{inputenc}
\usepackage[T1]{fontenc}
\usepackage{babel}
\usepackage{fourier}
\usepackage{listings}

\title{Typescript}

\begin{document}
\date{}

\maketitle

\section{Types}

\begin{description}
  \item[any]: set of all values, and you can do anything with it (makes your value behave like it would in regular JS)
  \item[unknown]: set of all values but supports only compare (\lstinline{==}, \lstinline{===}, \lstinline{||} \lstinline{&&} \lstinline{?}), negate (\lstinline{!}) and refine (\lstinline{typeof} and \lstinline{instanceof})
  \item[boolean]: two values true and \lstinline{false}. You can compare and negate them
  \item[number]: set of all numbers (integers, floats, positives, negatives, \lstinline{Infinity}, \lstinline{NaN}...) and all operations related to numbers
  \item[bigint]: set of large integers (without running into rounding errors)
  \item[string]: set of all strings and the things you can do with them like concatenate, slice and so on
  \item[symbol]: set of symbols (alternative to string keys in objects and maps)
  \item[object]: set of all objects but does not know anything about their structure (can't access any property)
  \item[object literal]: set of all objects with a specif shape. The thing might be an object literal or a class
  \item[Object]: this si pretty much the same as the empty object type \{\}, and it's best avoided
  \item[undefined]: the value \lstinline{undefined} is the only value with type \lstinline{undefined}
  \item[null]: same as undefined but with \lstinline{null}
  \item[void]: return type of a function that doesn't explicitely return anything
  \item[never]: return type of a function that never return at all (infinite loop of exception). It's a subtype of every other type so it's assignable to every other type.
\end{description}


\subsubsection{Remarks}

\begin{description}
  \item[const]: when declaring a variable, \lstinline{const} will narrow down the types. For example: \lstinline{let a = 10} will be infered as type 'number' but \lstinline{const a = 10} will be infered as type '10'. This behaviour only applies to primitive types (\lstinline{boolean, number, bigint, string, symbol}). Declaring an object with \lstinline{const} won't hint Ts to infer its type more narrowly because Js objects are mutable.
\end{description}

\subsection{Objects}

\subsubsection{Object Literals}

\begin{lstlisting}
  let a: {
    b: number,
    c?: string,
    [key: number]: boolean
  }
\end{lstlisting}

\begin{description}
  \item[Optional]: You can add the optional modifier \lstinline{?} to mark a property as optional
  \item[Index Signature]: 
    \lstinline{[key: T]: U} syntax is called an index signature. It's the way you tell Ts that the given object might contain more keys. The key's type \lstinline{T} must be assignable to either \lstinline{number} or \lstinline{string}.

    Note that you can use any word for the index signature key's name - it doesn't have to be key
  \item[Readonly]: You can mark fields as read-only with the \lstinline{readonly} modifier
\end{description}


\subsubsection*{Remarks}

\begin{description}
  \item[Empty object type]: object literal notation has one special case: empty object types (\{\}). Every type (except \lstinline{null} and \lstinline{undefined}) is assignable to an empty object type, which can make it tricky to use

\end{description}

\subsubsection{Arrays}

Arrays are a special kind of object that supports things like concatenation, pushing, searching and slicing. Ts supports two syntaxes for arrays: \lstinline{T[]} and \lstinline{Array<T>}. They are identical both in meaning and performance.

It's possible to create an inmutable array using the \lstinline{readonly} type: \\\lstinline{let a: readonly number[] = [1,2,3]}

\subsubsection{Tuples}

Tuples are subtypes of \lstinline{array}. Tuples have to be explicitely typed when you declare them (same syntax for tuples and arrays).

\subsubsection{Enums}

There are two types of enums: enums that map string to numbers and enums that map strings to strings

\section{Type operations}

\begin{description}
  \item[Type Aliases]: Just like any variable, it's possible to declare a type alias that points to a type: \lstinline{type Age = number} \\
    Type Aliases are block-scoped. Every block and every function has its own scope, and inner types aliases declarations shadow outer ones.
  \item[Union]: You can declare that something is of type A, B or both with the union operator \lstinline{A | B} 
  \item[Intersection]: You can declare that something is of type A and B with the intersection operator \lstinline{A & B} 
\end{description}

\section{Functions}

\begin{description}
  \item[Optional parameter]: \lstinline|(a: number, b?: number) => { ... }|
  \item[Default parameter]: \lstinline|(a: number, b: number = 10) => { ... }|. Notice how when we use a default value, we remove the optional annotation \lstinline{?}
  \item[Rest parameters]: \lstinline|(...numbers: number[]) => { ... }|
\end{description}

\subsubsection{Call Signatures}

Like the \lstinline{object} type, there is a \lstinline{Function} type. \lstinline{Function} is a catchall type for all functions, and doesn't tell anything about the specific function it types.

We can however also use call signatures: \lstinline{(a: number, b: number) => number}. Call signatures contain only type-level code (no values). That means they cannot express default values.

\subsubsection{Overloaded Function Types}

\begin{description}
  \item[call signature]: \lstinline{(a: number, b: number) => number}
  \item[Full call signature]:
    \begin{lstlisting}
type Add = {
  (a: number, b: number): number
  (a: number, b: number, c: number): number
}
    \end{lstlisting}
    Full call signature allow function overloading
\end{description}
\end{document}
