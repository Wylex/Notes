\documentclass[french]{article}

\usepackage[utf8]{inputenc}
\usepackage[T1]{fontenc}
\usepackage{babel}

\title{Pandas}

\begin{document}
\date{}

\maketitle

\setlength{\parindent}{0cm}

Library built on top of NumPy

\section{Load, display}

\begin{verbatim}
pd.read_csv(file_name)
df.head() / df.tail
df.shape
\end{verbatim}

Remove the `\ldots' when displaying:
\begin{verbatim}
pd.set_option(`display.max_columns', max_columns)
\end{verbatim}
Rq. Same for rows

\section{Dataframes}

A DataFrame can be seen as a dictionary where the keys are the columns and the values are the rows.\\
A DataFrame can be initialized with an actual dictionary:
\begin{verbatim}
df = pd.DataFrame(dic)
\end{verbatim}

\subsection{Columns}

How to acces a single column?
\begin{verbatim}
df[`column_name']
\end{verbatim}
Returns a Series object. It is a list of data but with more functionnalities.\\
It is also possible to use:
\begin{verbatim}
df.column_name
\end{verbatim}
Return a dataset subset (filter out columns):
\begin{verbatim}
df[[`column_name_1, `column_name_2']]
\end{verbatim}

\subsection{Rows}
Acces row by integer location (returns a Series):
\begin{verbatim}
df.iloc[0]
\end{verbatim}
Or for multiple rows (return a DataFrame):
\begin{verbatim}
df.iloc[[0,1]]
\end{verbatim}
And it's also possible to specify a column:
\begin{verbatim}
df.iloc[[0,1], [0,1]]
\end{verbatim}
It's possible to search rows by integers(iloc) but also by index:
\begin{verbatim}
df.loc[label]
df.loc[[labels_row], [labels_column]]
\end{verbatim}
Rq. By default row indexes are integer ids so \verb|iloc| $\Leftrightarrow$ \verb|loc|. \\

It is possible to pass slices (rows 0 to 10 included):
\begin{verbatim}
df.loc[0:10, label_1:label_n]
\end{verbatim}

\section{Indexes}

Indexes don't need to be unique in pandas but they usually are.\\
It's possible to change default indexes (integer id) with:
\begin{verbatim}
df.set_index(`column')
\end{verbatim}
It returns a new DataFrame with the index set to the column specified. Or we can use the argument \verb|inplace=True|.\\

To reset the indexes:
\begin{verbatim}
df.reset_index(inplace=True)
\end{verbatim}
It's possible to initialize a DataFrame with the optional argument:
\begin{verbatim}
index_col=`column'
\end{verbatim}
It's also possible to sort by indexes:
\begin{verbatim}
df.sort_index()
\end{verbatim}

\section{Filtering}

Filter mask:
\begin{verbatim}
filt = (df['column'] == 'something')
\end{verbatim}
Returns a Series object of True and False values. It's then possible to apply this filter:
\begin{verbatim}
df[filt]
\end{verbatim}
It will filter the DataFrame and will just keep the rows such that: \verb|row['column'] == x|\\
We get the same result with
\begin{verbatim}
df.loc[filt]
df.loc[filt, 'column']
\end{verbatim}

\subsection{filtering operators}

\& and | operators:
\begin{verbatim}
filt = (df['column'] == x) & (df['column2'] == y)
\end{verbatim}

Multiple conditions (equivalent notations):
\begin{verbatim}
countries = ['India', 'France', 'Spain']
filt = df['column'].isin(countries)

filt = (df['column'] == 'India') |
(df['column'] == 'France') |
(df['column'] == 'Spain')
\end{verbatim}

\subsection{Negate filter}
It's possible to get the opposite of the filter mask:
\begin{verbatim}
df.loc[~filt]
\end{verbatim}

\subsection{String methods}

\begin{verbatim}
filt = df['column'].str.contains('something')
filt = df['column'].str.contains('something', na=False)
\end{verbatim}
na parameter is for non string values (NaN or others)

\section{Update}

\subsection{Update columns}
We can update all the column names with:
\begin{verbatim}
df.columns = []
\end{verbatim}

Examples:
\begin{itemize}
\item To upperCase:
\begin{verbatim}
df.columns = [x.upper() for x in df.columns]
\end{verbatim}

\item Replace spaces:
\begin{verbatim}
df.columns = df.columns.str.replace(' ', '_')
\end{verbatim}
\end{itemize}

Update just some columns:
\begin{verbatim}
df.rename(columns={'oldName': 'newName'}, inplace=True)
\end{verbatim}

\subsection{Update rows}

Update everything:
\begin{verbatim}
df.iloc[0] = [c1, c2, ..., cn]
\end{verbatim}

We can specify columns:
\begin{verbatim}
df.iloc[0, [c1, c2]] = [c1, c2]
\end{verbatim}

Or single value:
\begin{verbatim}
df.iloc[0, c1] = c1
\end{verbatim}

Update a whole column (set column Series to another Series):
\begin{verbatim}
df['column'] = Series(something)
\end{verbatim}

\subsection{Tools}
\subsubsection{Apply}

Apply funtion on Series objects:
\begin{verbatim}
df.['email'].apply(len)
\end{verbatim}
Returns a Series object with the len of each email\\

Apply function on DataFrame:
\begin{verbatim}
df.apply(pd.Series.min)
\end{verbatim}
Applies the fucntion to each Series object (the columns). So this function returns a Series object with each column and the min function applied to each column.\\

It's possible to apply the function to rows instead of columns by changing the axis

\subsubsection{applymap (DataFrame)}
\begin{verbatim}
df.applymap(len)
\end{verbatim}

Applies the function to each element in the DataFrame. Returns a DataFrame

\subsubsection{map and replace (Series)}
\begin{verbatim}
df['c1'].map({'Yes': True, 'No': False})
\end{verbatim}
Returns a Series object replacing the values. Values not specified in the dictionary are replaced by NaN. The function \verb|replace| keeps these values

\section{Add and remove}

\subsection{Columns}

Add new column:
\begin{verbatim}
df['new_col'] = <Series>
\end{verbatim}

Remove column:
\begin{verbatim}
df.drop(columns=['c_name', 'c_name_2'], inplace=True)
\end{verbatim}

Split and create two new columns:
\begin{verbatim}
df[['c_name', 'c_name_2']] = df['c_name_3'].str.split(' ', expand=True)
\end{verbatim}
Rq. With \verb|expand=False|, \verb|split| returns a Series of lists. With \verb|expand=True| it returns a DataFrame with columns.

\subsection{Rows}

Add rows:
\begin{verbatim}
df.append({'column_name': 'value'})
\end{verbatim}
Rq. Missing keys will have a value of \verb|NaN|.


\section{Data preprocessing}

Vectorized string functions for Series and Index:
\begin{verbatim}
  <Series>.str
\end{verbatim}

\section{Tools}


\end{document}
