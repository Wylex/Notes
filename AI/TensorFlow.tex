\documentclass[french]{article}

\usepackage{listings}
\usepackage{color}
\usepackage{courier}
\usepackage[utf8]{inputenc}
\usepackage[T1]{fontenc}
\usepackage{babel}
\usepackage{bm}
\usepackage{amssymb}
\usepackage{amsthm}
\usepackage{booktabs}
\usepackage{amsmath}
\usepackage{mathrsfs}
\usepackage{verbatim}
\usepackage{graphicx}

\newcommand{\ra}[1]{\renewcommand{\arraystretch}{#1}}
\ra{1.3}

\title{TensorFlow}

\definecolor{dkgreen}{rgb}{0,0.6,0}
\definecolor{gray}{rgb}{0.5,0.5,0.5}
\definecolor{mauve}{rgb}{0.58,0,0.82}


\lstset{frame=tb,
  language=Python,
  aboveskip=3mm,
  belowskip=3mm,
  showstringspaces=false,
  columns=flexible,
  basicstyle={\small\ttfamily},
  numbers=left,
  frame=leftline,
  rulecolor=\color{gray},
  numberstyle=\tiny\color{gray},
  keywordstyle=\color{blue},
  commentstyle=\color{dkgreen},
  stringstyle=\color{mauve},
  breaklines=true,
  breakatwhitespace=true,
  tabsize=3,
  literate=
  {á}{{\'a}}1 {é}{{\'e}}1 {í}{{\'i}}1 {ó}{{\'o}}1 {ú}{{\'u}}1
  {Á}{{\'A}}1 {É}{{\'E}}1 {Í}{{\'I}}1 {Ó}{{\'O}}1 {Ú}{{\'U}}1
  {à}{{\`a }}1 {è}{{\`e}}1 {ì}{{\`i}}1 {ò}{{\`o}}1 {ù}{{\`u}}1
  {À}{{\`A}}1 {È}{{\'E}}1 {Ì}{{\`I}}1 {Ò}{{\`O}}1 {Ù}{{\`U}}1
  {ä}{{\"a}}1 {ë}{{\"e}}1 {ï}{{\"i}}1 {ö}{{\"o}}1 {ü}{{\"u}}1
  {Ä}{{\"A}}1 {Ë}{{\"E}}1 {Ï}{{\"I}}1 {Ö}{{\"O}}1 {Ü}{{\"U}}1
  {â}{{\^a}}1 {ê}{{\^e}}1 {î}{{\^i}}1 {ô}{{\^o}}1 {û}{{\^u}}1
  {Â}{{\^A}}1 {Ê}{{\^E}}1 {Î}{{\^I}}1 {Ô}{{\^O}}1 {Û}{{\^U}}1
  {œ}{{\oe}}1 {Œ}{{\OE}}1 {æ}{{\ae}}1 {Æ}{{\AE}}1 {ß}{{\ss}}1
  {ű}{{\H{u}}}1 {Ű}{{\H{U}}}1 {ő}{{\H{o}}}1 {Ő}{{\H{O}}}1
  {ç}{{\c c}}1 {Ç}{{\c C}}1 {ø}{{\o}}1 {å}{{\r a}}1 {Å}{{\r A}}1
  {€}{{\euro}}1 {£}{{\pounds}}1 {«}{{\guillemotleft}}1
  {»}{{\guillemotright}}1 {ñ}{{\~n}}1 {Ñ}{{\~N}}1 {¿}{{?`}}1
}

\begin{document}
\date{}

\maketitle

\section{Building Models}

\subsection{Simple models using the Sequential API}

The Sequential API is quite easy to use. However, although sequential models are extremely common, it is sometimes useful to build neural networks with more complex topologies, or with multiple inputs or outputs.

\begin{lstlisting}
model = keras.models.Sequential([
  keras.layers.Dense(30, activation="relu", input_shape=X_train.shape[1:]),
  keras.layers.Dense(1)
])

model.compile(loss="mean_squared_error", optimizer="sgd")
history = model.fit(X_train, y_train, epochs=20,
  validation_data=(X_valid, y_valid))
mse_test = model.evaluate(X_test, y_test)
\end{lstlisting}

\subsection{Complex models using the functional API}

With the functional API it's possible to have multiple inputs and outputs.

\begin{lstlisting}
input = keras.layers.Input(shape=X_train.shape[1:])
hidden1 = keras.layers.Dense(30, activation="relu")(input)
hidden2 = keras.layers.Dense(30, activation="relu")(hidden1)
concat = keras.layers.Concatenate()[input, hidden2])
output = keras.layers.Dense(1)(concat)
model = keras.models.Model(inputs=[input], outputs=[output])
\end{lstlisting}

\begin{figure}[h]
\includegraphics[scale=0.1]{Wide and Deep NN}
\centering
\end{figure}

\subsection{Dynamic models using the subclassing API}

Both the Sequential API and the Functional API are declarative: you start by declaring which layers you want to use and how they should be connected, and only then can you start feeding the model some data for training or inference. This has many advantages (saving, cloning, sharing, the structure can be analized, tf can infer shapes, check types...) but the flip side is just that: it’s static.

Some models involve loops, varying shapes, conditional branching, and other dynamic behaviors. For such cases, or simply if you prefer a more imperative programming style, the Subclassing API is for you.

\begin{lstlisting}
class WideAndDeepModel(keras.models.Model):
  def __init__(self, units=30, activation="relu", **kwargs):
    super().__init__(**kwargs) # handles standard args (e.g., name)
    self.hidden1 = keras.layers.Dense(units, activation=activation)
    self.hidden2 = keras.layers.Dense(units, activation=activation)
    self.main_output = keras.layers.Dense(1)
    self.aux_output = keras.layers.Dense(1)

  def call(self, inputs):
    input_A, input_B = inputs
    hidden1 = self.hidden1(input_B)
    hidden2 = self.hidden2(hidden1)
    concat = keras.layers.concatenate([input_A, hidden2])
    main_output = self.main_output(concat)
    aux_output = self.aux_output(hidden2)
    return main_output, aux_output

model = WideAndDeepModel()
\end{lstlisting}

This example looks very much like the Functional API, except we do not need to create the inputs, we just use the input argument to the \lstinline{call()} method, and we separate the creation of the layers in the constructor from their usage in the \lstinline{call()} method.  However, the big difference is that you can do pretty much anything you want in the \lstinline{call()} method: for loops, if statements, low-level TensorFlow operations... This makes it a great API for researchers experimenting with new ideas.

This extra flexibility comes at a cost: your model’s architecture is hidden within the \lstinline{call()} method, so Keras cannot easily inspect it, it cannot save or clone it, and when you call the \lstinline{summary()} method, you only get a list of layers, without any information on how they are connected to each other. Moreover, Keras cannot check types and shapes ahead of time, and it is easier to make mistakes. So unless you really need that extra flexibility, you should probably stick to the Sequential API or the Functional API.

\section{Computing Gradients with Autodiff}

\section{Loading and Prepocessing Data}

TensorFlow takes care of all implementation details, such as multithreading, queuing, batching, prefetching, and so on.

\subsection{The Data API}

The whole Data API revolves around the concept of a dataset: this represents a sequence of data items. Usually you will use datasets that gradually read data from disk.

Once you have a dataset, you can apply all sorts of transformations to it by calling its transformation methods. Each method returns a new dataset, so you can chain transformations.

\end{document}

