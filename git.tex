\documentclass[french]{article}

\usepackage[utf8]{inputenc}
\usepackage[T1]{fontenc}
\usepackage{babel}
\usepackage{fourier}
\usepackage{listings}

\title{Git}

\begin{document}
\date{}

\maketitle

\setlength{\parindent}{0cm}

\section{Git setup}

Git config operates at three levels:
\begin{enumerate}
    \item \emph{/etc/gitconfig}: Values for every user on the system \lstinline[language=bash]{git config --system}
    \item \emph{~/.gitconfig} or \emph{~/.config/git/config}: Specific to the user \lstinline[language=bash]{git config --global}
    \item \emph{config}: Specific to the repository
\end{enumerate}
Each level overrides values in the previous level

\begin{description}
    \item[Indentity]: Every git commit uses the name and email:
        \begin{enumerate}
            \item[-] \lstinline{git config --global user.name "John Doe"}
            \item[-] \lstinline{git config --global user.email "johndoe@example.com"}
        \end{enumerate}
    \item[Editor]: Every git commit uses the name and email:
        \begin{enumerate}
            \item[-] \lstinline{git config --global core.editor emacs}
        \end{enumerate}
    \item[Checking your settings]: Every git commit uses the name and email:
        \begin{enumerate}
            \item[-] \lstinline{git config --list}
        \end{enumerate}
    \item[Getting help]:
        \begin{enumerate}
            \item[-] \lstinline{git help <verb>}
            \item[-] \lstinline{git <verb> --help}
            \item[-] \lstinline{man git-<verb>}
        \end{enumerate}
\end{description}

\subsection{Aliases}

\begin{description}
    \item[] \lstinline{git config --global alias.co checkout}
    \item[] \lstinline{git config --global alias.br branch}
    \item[] \lstinline{git config --global alias.ci commit}
    \item[] \lstinline{git config --global alias.st status}
\end{description}


\section{Basics}

\begin{description}
    \item[Initialize repo]:
        \begin{enumerate}
            \item[-] \lstinline{git init}: creates a new subdirectory \emph{.git}
            \item[-] \lstinline{git clone <url> [name]}: copy existing repository. Every version of every file is pulled down by default
        \end{enumerate}
    \item[Recording changes]:
        \begin{enumerate}
            \item[-] \lstinline{git status}: main tool to determine which files are in which state
        \end{enumerate}
    \item[Viewing staged and unstaged changes]:
        \begin{enumerate}
            \item[-] \lstinline{git diff}: similar to git status but shows the exact lines added and removed. It compares your working directory with your staging area. If you want to see what's staged that will go into the next commit, you can use \lstinline{gid diff --staged}
        \end{enumerate}
    \item[Tracking new files]:
        \begin{enumerate}
            \item[-] \lstinline{git add <file or directory>}: begin tracking a new file, stage files, marking merge-confliected files as resolved... \\
                Git stages a file exactly as it is when you run \lstinline{git add}. If you modify the file, the new change won't be staged.
        \end{enumerate}
    \item[Ignoring files]:
        \begin{enumerate}
            \item[-] If you don't want a file to show as untracked, you can include it in the \emph{.gitignore} file.
        \end{enumerate}
\end{description}

\subsection{Commiting Your Changes}

\begin{description}
    \item[Commiting changes]:
        \begin{enumerate}
            \item[-] \lstinline{git commit}: once your staging are is set up the way you want it, you can commit.
        \end{enumerate}
    \item[Removing files]:
        \begin{enumerate}
            \item[-] \lstinline{git rm}: remove a file from the tracked files and also removes the file from your working directory.
            \item[-] \lstinline{git rm -f}: If you added the file create and add the file to the index, you must use \lstinline{-f} option. It prevents accidental removal of data that hasn't yet been recorded in a snapshot.
            \item[-] \lstinline{git rm --cached}: Keep a file in the working tree but remove it from the staging area.
        \end{enumerate}
    \item[Moving files]:
        \begin{enumerate}
            \item[-] \lstinline{git mv <from> <to>}: Rename a file.
        \end{enumerate}
\end{description}

\subsection{Viewing the commit history}

\begin{description}
    \item[View history]:
        \begin{enumerate}
            \item[-] \lstinline{git log}: Some of the most popular options are
                \begin{enumerate}
                    \item \lstinline{-p} which shows the difference introduced in each commit
                    \item \lstinline{-<number>} limit the number of entries
                    \item \lstinline{--pretty} change the log output formats
                    \item \lstinline{--graph} add an ASCII graph showing your branch and merge history
                    \item \lstinline{--since=2.weeks} list of commits made in the last two weeks
                    \item \lstinline{--S<string>} filter commits that added or removed a reference to a specific location
                    \item \lstinline{-- <path>} filter commits that modified a file
                \end{enumerate}
        \end{enumerate}
\end{description}

\subsection{Undoing Things}

\begin{description}
    \item[Modify last commit]:
        \item[-] \lstinline{git commit --amend}: Take the staging area and uses it for the commit. It overwrites the previous commit.
    \item[Unstaging a staged file]:
        \item[-] \lstinline{git reset HEAD <filename>}
    \item[Discard changes]:
        \item[-] \lstinline{git checkout -- <filename>}
\end{description}

\subsection{Working with remotes}

\begin{description}
    \item[List remotes]: \lstinline{git remote}: See which remotes are configured. If you cloned the repository, you should at least see origin (default name Git gives to the server you cloned from). You can use \lstinline{-v} to see the url
    \item[Add remotes]: \lstinline{git remote add <shortname> <url>}
    \item[Fetch]: \lstinline{git fetch <remote name>}: pulls down all the data from that remote project that you don't have yet.
    \item[Pull]: \lstinline{git pull}: if the branch is set up to track a remote branch, you can pull to automatically fetch and merge into the current branch.
    \item[Push]: \lstinline{git push <remote> <branch>}
    \item[Removing]: \lstinline{git remote rename <name> <new name>}
    \item[Removing remote] \lstinline{git remote rm <name>}
\end{description}

\subsection{Tagging}

\begin{description}
    \item[List tags] \lstinline{git tag}
    \item[Creating tags (annotated)] \lstinline{git tag -a <name> -m <message> [commit checksum]}
    \item[Show tag info] \lstinline{git show <tag>}
    \item[Creating tags (lightweight)] \lstinline{git tag <name> [commit checksum]}: this is basically the commit checksum stored in a file
    \item[Sharing tags] \lstinline{git push <remote> <tag>}: \lstinline{git push} does not push tags to the remote server
\end{description}

\section{Git Branching}


\section{Git flow}
Each file in your working directory can be in two states: tracked or untracked. Tracked files are files that were in the last snapshot. They can be unmodified, modified or staged. Untracked files are everything else.

\section{Git}


View status:
\begin{verbatim}
  git status
\end{verbatim}

Add to snapshot:
\begin{verbatim}
  git add <file>
\end{verbatim}

Commit snapshot:
\begin{verbatim}
  git commit
\end{verbatim}

View a repository's commit history:
\begin{verbatim}
  git log
  git log -n <number-of-commits>
\end{verbatim}

Show commits in branch-name2 that aren't in branch-name:
\begin{verbatim}
  git log <branch-name>..<branch-name2>
\end{verbatim}

Define author name and email:
\begin{verbatim}
  git config user.name
  git config user.email
\end{verbatim}

\section{Revert changes}

View previous commit (moves the HEAD):
\begin{verbatim}
  git checkout <commit-id>
\end{verbatim}
Rq. The HEAD is Git's internal way of indicating the snapshot that is currently checked out. The HEAD normally resides on the tip of a development branch. When we checkout a previous commit we can no longer say we are on the <branch-name> branch since it contains more recent snapshots than the HEAD $\rightarrow$ currently on no branch.\\

Go back to last commit:
\begin{verbatim}
  git checkout master
\end{verbatim}

Tag a commit (for easier acces):
\begin{verbatim}
  git tag -a <tag-name> -m <description>
\end{verbatim}

Revert a commit (commits a new snapshot without the specified one):
\begin{verbatim}
  git revert <commit-id>
\end{verbatim}

Unstage files:
\begin{verbatim}
  git reset
\end{verbatim}

Undo uncommitted changes (\danger\ operate on the working directory, not on committed snapshots. Permanently undo changes):
\begin{itemize}
\item [-] Change all tracked files to match the most recent commit:
\begin{verbatim}
  git reset --hard
\end{verbatim}
\item [-] Remove all untracked files:
\begin{verbatim}
  git clean -f
\end{verbatim}
\end{itemize}

\section{Branches}

Rq. It's not possible to add new commits when not on a branch.\\

List existing branches:
\begin{verbatim}
  git branch
\end{verbatim}

Create new branch:
\begin{verbatim}
  git branch <branch-name>
\end{verbatim}

Checkout branch:
\begin{verbatim}
  git checkout <branch-name>
\end{verbatim}

Create and checkout branch:
\begin{verbatim}
  git checkout -b <branch-name>
\end{verbatim}

Remove file from working directory and index:
\begin{verbatim}
  git rm <file-name>
\end{verbatim}

Merge branch to current one:
\begin{verbatim}
  git merge <branch-name>
\end{verbatim}

Rq. Merge types
\begin{itemize}
  \item [-] Fast-forward merge: There's no commits since merged branch was created. Git simply moves the tip of the branch and doesn't need a merge commit.
  \item [-] 3-way merge: Two branches whose history has diverged. It creates an extra commit.
\end{itemize}$ $\\

Delete branch
\begin{verbatim}
  git branch -d <branch-name>
\end{verbatim}

\section{Rebasing}

Rebase branch to current branch:
\begin{verbatim}
  git rebase <new-base>
\end{verbatim}

Interactive rebase (Rebase and modify commits):
\begin{verbatim}
  git rebase -i <new-base>
\end{verbatim}

Rq. Interactive rebase commands:
\begin{itemize}
  \item [-] pick: keep that commit
  \item [-] squash: combine commit with previous one
  \item [-] edit: gives the opportunitu to alter the staged snapshot before committing it. \verb|git rebase --amend| to add changes to commit and \verb|git rebase --continue| to continue rebase
\end{itemize}
If lost in the middle of a rebase and afraid to continue use \verb|git rebase --abort| to abandon it and start over from scratch.
$ $\\

\section{Rewriting history}

Split commits:
\begin{verbatim}
  git rebase -i <base>
\end{verbatim}
Then use the \verb|edit| command on the commit you want to split.\\
The repository history will be set just after that commit.\\
Instead of \verb|git rebase --amend| that would allow to modify the commited snapshot, the commit is completely removed with: \\ \verb|git reset --mixed HEAD|\textasciitilde \verb|1|
\begin{itemize}
  \item [-] The \verb|HEAD|\textasciitilde \verb|1| parameter tells it to reset to the commit that occurs immediately before the current HEAD
  \item [-] The \verb|-- mixed| flag preserves the working directory
\end{itemize}
It's then possible to make as many commits as possible and finish with \verb|git rebase --continue|\\

View every change made in the repository:
\begin{verbatim}
  git reflog
\end{verbatim}
Rq. It's possible with it to find dangling commits that would otherwise be lost from the project history.

\section{Remotes}

First off, configure repository:
\begin{verbatim}
  git config user.name
  git config user.email
\end{verbatim}

List the connections to other repositories:
\begin{verbatim}
  git remote
  git remote -v
\end{verbatim}
Rq. When you clone a repository, Git automatically adds an origin remote pointing to the original repository. \\

Add remote:
\begin{verbatim}
  git remote add <remote>
\end{verbatim}

Fetch remote branchs:
\begin{verbatim}
  git fetch <remote>
\end{verbatim}
Rq. Fetched branches are renamed in the local repository as <remote-name>/<branch>.\\

View remote branchs:
\begin{verbatim}
  git branch -r
\end{verbatim}

Push branch to remote repository:
\begin{verbatim}
  git push <remote> <branch>
\end{verbatim}
Rq. \danger Pushed branches become local branches in the remote repository.\\

Tags must be pushed by their own:
\begin{verbatim}
  git push <remote> <tag>
\end{verbatim}

\section{Centralized workflows}

Create central repository:
\begin{verbatim}
  git init --bare <directory-name>
\end{verbatim}
The \verb|--bare| flag tells Git that we don't want a working directory. This will prevent us from developing in the central repository. It is a convention to have a .git extension in the directory name. \\

We then have to add the remote:
\begin{verbatim}
  git add remote origin <remote>
\end{verbatim}


We can then push our branches:
\begin{verbatim}
  git push origin master
\end{verbatim}

Rq. Git won't let anyone push to a remote server if it doesn't result in a fast-forward merge.\\

Rq. Never, ever rebase commits that have been pushed to a shared repository. If you need to change a public commit, use the \verb|git revert| command.

\section{Distributed workflows}

Using the integrator workflow our private development process largely ramains the same but we've added an additional task: incorporating changes from third-party contributors.\\
You'll never have to worry about security using an integrator workflow because you'll still be the only one with access to the ``official'' repository.\\
There's also an interesting side effect to this kind of security. By giving each developer their own public repository, the integrator workflow creates a more stable development environment for open-source software projects. Should the lead developer stop maintaining the ``official'' repository, any of the other participants could take over by simply designating their public repository as the new ``official'' project.

\section{Patched workflows}

Create a patch from commit:
\begin{verbatim}
  git format-patch <branch>
\end{verbatim}
The branch parameter tells Git to generate patches for every commit in the current branch that's missing from <branch>\\
This will generate .patch files that contains enough information to re-create the commit from the last step.\\

Once the patches are generated, it's possible to send them via email. It's possible to do that with the git command \verb|git send-email .| which sends all patches in the ./ directory.\\

It's then possible to apply the patches with \verb|git am < path-file|\\
\verb|git am| reads from standard imput and generates the commit.\\

In many ways, patches are a simpler way to accept contributions than the integrator workflow (using branches). Only the project maintainer needs a public repository, and he'll never have to peek at anyone else's repository.

\section{Miscellaneous}

Stash uncommitted changes:
\begin{verbatim}
  git stash
\end{verbatim}

Reapply stashed changes:
\begin{verbatim}
  git stash apply
\end{verbatim}
Rq. It's possible to apply a stash to any branch, not just the one from which it was created. So for example if you find yourself yourself developing on the wrong branch, you could stash all your changes, and apply them to the correct one.\\

View changes:
\begin{verbatim}
  git diff <branch>..<branch2>
\end{verbatim}

Can be used to generate a detailed view of uncommitted changes:
\begin{verbatim}
  git diff
\end{verbatim}

Revert one file to the specified commit:
\begin{verbatim}
  git checkout HEAD <file>
\end{verbatim}

It's possible to define aliases:
\begin{verbatim}
  git config --global alias.co checkout
\end{verbatim}

\section{Plumbing}

View git representation of a commit:
\begin{verbatim}
  git cat-file commit <commit>
\end{verbatim}

Git representation of a commit:
\begin{itemize}
  \item [-] a tree
  \item [-] a parent
  \item [-] user data
  \item [-] a message
\end{itemize}
$ $\\
A tree object is Git's representation of the ``snapshots''. They record the state of a directory at a given point, without any notion of time or author.\\
To tie trees together into a coherent project history, Git wraps each one in a commit object and specifies a parent, whitch is just another commit.\\

View tree representation:
\begin{verbatim}
  git ls-tree <tree-id>
\end{verbatim}
Rq. ``blobs'' represent files and ``trees'' represent folders. Blob objects are how Git stores file data and tree objects combine blobs and other trees into a directory listing.\\

View blob representation:
\begin{verbatim}
  git cat-file blob <blob>
\end{verbatim}
This should display the entire contents of the file. Blobs are pure content: there is no mention of a filename. The name is stored in the tree that contains the blob not the blob itself.\\

The fourth and last type of Git object is the tag object:
\begin{verbatim}
  git cat-file tag <tag>
\end{verbatim}

Rq. A branch is just a reference to a commit object.

\end{document}
